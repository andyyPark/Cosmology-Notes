\subsection{Discrete tracers}
The theory can be applied to discrete tracers, such as galaxies, with some small modifications. Modifications are necessary for two reasons: the first is that there can be a non-negligible scatter in the number of tracers that occupy haloes of the same mass; the second is that when computing the autocorrelation of a discrete tracer field, there is an automatic correlation of the field with itself at zero separation, the so-called shot noise. In configuration space, this manifests at $r=0$ in the correlation function, but in Fourier space this is spread evenly over all wavenumber, resulting in a constant shot noise power spectrum, $P^{sn} = 1/\Bar{n}$, where $\Bar{n}$ is the mean tracer number density.

Consider a field of galaxies and in particular the galaxy number density contrast, $\theta_u = \delta_g = (n_g - \Bar{n}_g)/\Bar{b}_g$. The mean number density of galaxies is defined as
\begin{equation}
    \Bar{n}_g = \langle n_g \rangle = \int_0^\infty dM n(M) N_g(M).
\end{equation}
