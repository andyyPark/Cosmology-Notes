

\subsection{Introduction}

In early times, when the matter fluctuations in the universe were small, linear perturbation theory can be used to describe the large scales. However, a full non-linear treatment is needed as structures form and fluctuations become larger. The halo model tries to approximate the matter distribution in the non-linear regime. It assumes that all matter resides in haloes, which are $O(100)$ times denser than that of numerical simulations. Once we know the properties and distribution of these haloes, we can use estimate the statistical properties of the matter distribution, hence it can be used in cosmological analysis. 

The statistical properties of any tracer of matter can be modeled, provided that the connection between the tracer and host haloes is known. 

\begin{principle}
If haloes are taken to be the host of galaxy formation, all that is needed to model the galaxy clustering signal is how galaxies occupy haloes of different masses. The problem can be split into how galaxies cluster within the same halo and how different haloes cluster with respect to each other.
\end{principle}

The halo properties needed for a halo model are the halo bias (how haloes cluster relative to matter), halo mass function (number density of haloes with different masses), and halo profile (how matter or its tracers are distributed within a halo). These ingredients are mostly extracted from numerical simulations and calibrated across a range of cosmological parameters. It is usual to assume that haloes are linearly biased, spherical objects with properties that are only a function of the halo mass. We call this method of using the halo model the ``analytical approach".

The second approach to the halo model is the ``simulation-based approach''. Here, haloes are identified in a simulation and then `painted' with a specific tracer (e.g., galaxies), such that the desired clustering properties can be directly measured. Which analytical approach is faster and more flexible, the simulation-based approach is more accurate but is slower and requires $N$-body simulations.

The halo model has been used in one form or another to analyze data from weak gravitational lensing by large-scale structures. This is because cosmic shear relies on information from non-linear matter distribution. Data from galaxy-galaxy lensing have been analyzed with a flexible halo model to capture information from smaller scales. The halo model can also predict the intrinsic alignments of galaxies. 

The halo model should be calibrated against simulations to check its accuracy, and a failure to do so may result in incorrect parameter constraints where the real signal is mistaken for some modeling deficiency. 