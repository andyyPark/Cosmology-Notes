\subsection{Introduction}

The Gaussian likelihood is sufficient to obtain accurate cosmological results from weak lensing pseudo-$C_\ell$ estimates. An important ingredient for a Gaussian likelihood is the covariance matrix. The problem of calculating covariance matrices for cosmic shear has been extensively discussed in the literature, ranging from analytic or semi-analytic approaches to estimation from simulation.

The covariance matrix of lensing two-point statistics can be organized into three physically distinct types of contributions; the Gaussian (G), super-sample covariance (SSC), and connected non-Gaussian (cNG) terms. The G term is the minimal covariance contribution and it would be the only contribution to the covariance if the noisy shear field itself was Gaussian distributed; this is approximately correct on sufficiently large scales (multiples $\ell < 100-200$ for galaxy source redshifts $z_s\approx1$). The SSC term describes the correlation between the two-point function on different scales that are induced by large-scale density/tidal fluctuations in which the entire surveyed region is embedded. Finally, the cNG term describes the contribution to the covariance that is induced by nonlinear structure formation within the survey volume, i.e., when the density fluctuations grow to order unity and the field becomes appreciable non-Gaussian distributed. The Gaussian term can be calculated given the survey footprint and the nonlinear matter power spectrum. The SSC term can also be fully specified by the survey footprint and the power spectrum. The cNG term is controlled by the so-called parallelogram configuration of the nonlinear matter trispectrum. 