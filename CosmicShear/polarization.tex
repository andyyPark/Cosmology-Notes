\subsection{CMB Polarization}

This subsection talks about the polarized CMB, but we can take over all results from this section to cosmic shear.

The polarization of a radiation field can be measured by inserting a polarizer in front of the detector, which allows only waves oscillating in a particular direction (in the plane perpendicular to propagation) to pass through. By plotting the intensity recorded by the detector as a function of the orientation of the polarizer, one can measure the polarization. If you rotate the polarizer by 180\textdegree, you get the same result, thus polarization is a \textit{headless vector}. Let $\bm{\hat{m}}$ be the unit direction vector of the polarizer (where $\bm{\hat{m}} \cdot \bm{\hat{p}} = 0$, with $\bm{\hat{p}}$ being the vector of the radiation), the flux of radiation incident on the detector cannot depend on the sign of $\bm{\hat{m}}$. Therefore it must be a quadratic function of $\bm{\hat{m}}$. Define
\begin{equation}
    \label{eq:pol_inten}
    I_\text{det}(\bm{\hat{m}}) = I_{ij}\bm{\hat{m}}^i\bm{\hat{m}}^j,
\end{equation}
where $I_{ij}$ is the polarization tensor. For unpolarized light, the detected intensity is identical in the $\bm{\hat{m}}_x$ and $\bm{\hat{m}}_y$ directions, so $I_{ij} \propto \delta_{ij}$. Write
\begin{equation}
    \label{eq:pol_tensor}
    I_{ij} = \begin{pmatrix}
        I + Q & U \\
        U & I - Q
    \end{pmatrix}.
\end{equation}

The diagonal elements $I$ are the intensity (temperature $T$ for CMB). The two new variables $Q$ and $U$ describe polarization. $I, Q$, and $U$ are the three out of the four Stokes parameters used in classic electromagnetism. Circular polarization is not generated by cosmological perturbation, so we ignore $V$.

Assume flat-sky approximation and write equation \ref{eq:pol_tensor} as
\begin{equation}
    \label{eq:pol_tensor2}
    I_{ij} = I\delta_{ij} + I^T_{ij},
\end{equation}
where $I^T_{ij}$ is traceless and contains the information needed about the two polarization states. Instead of $Q$ and $U$, we want to characterize those two states by their behavior under rotations. Decompose $I^T_{ij}$ into scalar, vector, and tensor perturbations. Let $E(\bm{\ell})$ be the scalar decomposition with $\ell_i\ell_jI^T_{ij}/\ell^2$ and $I^{TT}_{ij}$ be the transverse-traceless tensor (so $\ell^iI^{TT}_{ij}=0$). Then
\begin{equation}
    \label{eq:pol_tensor3}
    I^T_{ij} = 2 \left(\frac{\ell_i \ell_j}{\ell^2} - \frac{1}{2}\delta_{ij}\right)E(\bm{\ell}) + I^{TT}_{ij}.
\end{equation}

Solving for $E(\bm{\ell})$ gives 
\begin{align}
    \nonumber E(\bm{\ell}) &= \frac{\ell^i \ell^j}{\ell^2}I^T_{ij} \\
    \nonumber &= (\frac{\ell_x^2}{\ell^2} - \frac{\ell_y^2}{\ell^2})Q(\bm{\ell}) + 2 \frac{\ell_x\ell_y}{\ell^2}U(\bm{\ell}) \\
    &= (\cos^2\phi_\ell - \sin^2\phi_\ell)Q(\bm{\ell}) + 2\sin\phi_\ell\cos\phi_\ell U(\bm{\ell}) \\
    \label{eq:pol_emode}
    &= \cos2\phi_\ell~ Q(\bm{\ell}) + \sin2\phi_\ell~ U(\bm{\ell}).
\end{align}

Use equation \ref{eq:pol_tensor3} to find $I^{TT}_{ij}$. First,
\begin{align}
    \nonumber I^{TT}_{xy} &= I^T_{xy} - 2\frac{\ell_x\ell_y}{\ell^2} E(\bm{\ell}) \\
    \nonumber &= U(\bm{\ell}) - \sin2\phi_\ell (\cos2\phi_\ell~ Q(\bm{\ell}) + \sin2\phi_\ell~ U(\bm{\ell}).) \\
    \nonumber &= (1 - \sin^2 2\phi_\ell)U(\bm{\ell}) - \sin 2\phi_\ell \cos 2\phi_\ell Q(\bm{\ell}) \\
    &= \cos 2\phi_\ell B(\bm{\ell}),
\end{align}
where
\begin{equation}
    \label{eq:pol_bmode}
    B(\bm{\ell}) = -\sin 2\phi_\ell ~Q(\bm{\ell}) + \cos 2\phi_\ell ~U(\bm{\ell}).
\end{equation}

Finding the other components of $I^{TT}_{ij}$, we can write the traceless tensor as
\begin{equation}
    \label{eq:pol_tensor4}
    I^T_{ij}(\bm{\ell}) = \begin{pmatrix}
        \cos2\phi_\ell & \sin2\phi_\ell \\
        \sin2\phi_\ell & -\cos2\phi_\ell
    \end{pmatrix} E(\bm{\ell}) + \begin{pmatrix}
        -\sin2\phi_\ell & \cos2\phi_\ell \\
        \cos2\phi_\ell & \sin2\phi_\ell
    \end{pmatrix} B(\bm{\ell}).
\end{equation}
The $E$-mode varies in strength in the same direction as, or perpendicular to, its orientation. This conjures images of an electric field. The $B$-mode varies in strength in a different direction from that in which it is pointing (by 45\textdegree) just like a magnetic field.
